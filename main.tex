\documentclass{article}
\usepackage[utf8]{inputenc}

\title{CS 6000 Journal 3}
\author{David Stout}
\date{September 15 2018}

\usepackage{natbib}
\usepackage{graphicx}

\begin{document}

\maketitle

\section{Learning to Write a Survey Paper}
Approaching this paper was rather daunting, I have never written anything like it and was very unfamiliar with the 
format. The first thing that I have done to increase my confidence and start down the process of learning how to write a 
survey paper was to watch the lecture video over again. I find that there is so much information that is presented in 
class that I may miss a thing or two, and the lecture videos are very helpful in filling in the gaps.

After watching the lecture a few times, I thought that the next course of action for me was to speak with my PhD adviser.
I scheduled a meeting with Dr. Zhuang and discussed what a survey paper is and how to go about approaching it. She was 
able to give me links to two separate survey papers that I could scan to get a better idea of what a survey paper is. I 
thought that it was very helpful to see one in full to really put together what a survey paper is, how it is structured, 
and the approach that is used to put together the true idea behind the topic. We also spoke about using this paper as my 
oral examination next semester and what that would entail. 

I set to work finding sources about my topic using Google scholar, as was discussed in class, and I found it easy to find
papers that were relevant to my topic. My main issue that popped up when using Google scholar, was "falling down the 
rabbit hole". I found that there are a lot of interesting papers that are not very relevant to my topic so I had to 
really try to limit my time spent on each paper to just determine relevance and not my own personal interest. I got 
locked out of Google scholar because I was adding too many sources into Zotero too quickly, luckily shortly after this 
happened I received the notification from Canvas that it also happened to Dr. Boult and that it was a temporary thing. 
This helped to alleviate my panic that I was no longer able to use the most useful resource we have available for this 
paper.

Moving forward I plan to continue to gather sources and really refine my topic. I believe that I will be able to start 
the main body of my paper tomorrow after work, and will continue gaining more confidence in my ability to do this project
well.
\end{document}
